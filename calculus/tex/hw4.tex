\documentclass{article}
\usepackage[utf8]{inputenc}
\usepackage{amsmath, amssymb}

\title{DSBA Calculus HW4}
\author{Kirill Korolev, 203-1}
\date{2th of October, 2020}

\begin{document}

\maketitle

\begin{enumerate}

\item Find the following limits:

\begin{enumerate}
\item $\lim_{n \to \infty} (\sqrt{n^2 + 1} - \sqrt{n^2 - 1}) \cdot \sin{(n^2 + 3)}$


Because $-1 \leq sin(n^2 + 3) \leq 1$
\begin{align*}
&x_n = -(\sqrt{n^2 + 1} - \sqrt{n^2 - 1}) \leq (\sqrt{n^2 + 1} - \sqrt{n^2 - 1}) \cdot \sin{(n^2 + 3)}\\
&(\sqrt{n^2 + 1} - \sqrt{n^2 - 1}) \cdot \sin{(n^2 + 3)} \leq (\sqrt{n^2 + 1} - \sqrt{n^2 - 1}) = z_n\\
&\lim_{n \to \infty} x_n = \lim_{n \to \infty} -(\sqrt{n^2 + 1} - \sqrt{n^2 - 1}) = -\lim_{n \to \infty} \frac{2}{\sqrt{n^2 + 1} + \sqrt{n^2 - 1}} = 0\\
&\lim_{n \to \infty} z_n = \lim_{n \to \infty} (\sqrt{n^2 + 1} - \sqrt{n^2 - 1}) = \lim_{n \to \infty} \frac{2}{\sqrt{n^2 + 1} + \sqrt{n^2 - 1}} = 0\\
\end{align*}

Then by sandwich theorem:
\[\lim_{n \to \infty} (\sqrt{n^2 + 1} - \sqrt{n^2 - 1}) \cdot \sin{(n^2 + 3)}=0\]

\item $\lim_{n \to \infty} \frac{(4\cos{n} - 3n)^2(2n^5 - n^3 + 1)}{(6n^3 + 5n\sin{n})(n + 2)^4}$

If we multiply inf. small by bounded sequence, we will get inf. small again. 

\begin{align*}
\lim_{n \to \infty} \frac{(4\cos{n} - 3n)^2(2n^5 - n^3 + 1)}{(6n^3 + 5n\sin{n})(n + 2)^4} = \lim_{n \to \infty} \frac{n^2(\frac{4\cos{n}}{n} - 3)^2n^5(2 - \frac{1}{n^2} + \frac{1}{n^5})}{n^3(6 + \frac{5\sin{n}}{n^2})n^4(1 + \frac{2}{n})^4} = \lim_{n \to \infty} \frac{3^2 \cdot 2}{6 \cdot 1} = 3
\end{align*}

\end{enumerate}

\item Prove that the following sequences are the Cauchy sequences:

\begin{enumerate}
\item $x_n = 0.77...7$ (n digits)

\begin{align*}
\forall \epsilon > 0, \exists N \in \mathbb{N}: \forall n \geq N, \forall p > 0 \:\: |\underbrace{0.77..77}_\text{n + p 7's} - \underbrace{0.77..77}_\text{n 7's}| < \epsilon\\
|0.77..77 - 0.77..7| = |0.\underbrace{000...}_\text{n digits}\underbrace{777...}_\text{p digits}|
\end{align*}

For $\epsilon \geq 1$ inequality is obvious, suppose $0 < \epsilon < 1$, then we can represent $\epsilon$ as:

\[\epsilon = 0.a_{1}a_{2}...a_{m}\] where $a_i \in \{0, ..., 9\}, \forall i \in [m]$

Then starting from the left we can find first non-zero digit $a_j$. 

Afterwards we choose $N = j + 1$ so that:
\[|0.\underbrace{000}_\text{j + 1 digits}777..| < 0.\underbrace{000}_\text{j - 1 digits}a_j....a_m\]

Therefore $\{x_n\}$ is a Cauchy sequence.

\end{enumerate}

\item Prove that the sequence converges:
\begin{enumerate}
\item $\sum_{k = 1}^n \frac{\cos{ka}}{2^k}$

It is equivalent to prove that sequence is a Cauchy sequence. By using the definition written above:

\begin{align*}
&\bigg|\frac{\cos{1a}}{2^1} + ... + \frac{\cos{(n + p)}}{2^{n+p}} - \frac{\cos{1a}}{2^1} - ... - \frac{\cos{n}}{2^n}\bigg| = \bigg|\frac{\cos{(n + 1)}}{2^{n+1}} + ... + \frac{\cos{(n + p)}}{2^{n+p}}\bigg| \leq\\
&\bigg|\frac{\cos{(n + 1)}}{2^{n+1}}\bigg| + ... + \bigg|\frac{\cos{(n + p)}}{2^{n+p}}\bigg| \leq \frac{1}{2^{n+1}} + ... + \frac{1}{2^{n+p}} = \frac{1}{2^n} \bigg(\frac{1}{2} + ... + \frac{1}{2^p}\bigg) =\\
&\frac{1}{2^n} \cdot \frac{1}{2} \cdot \frac{(\frac{1}{2})^n - 1}{\frac{1}{2} - 1} = \frac{1}{2^n} \cdot (1 - (\frac{1}{2})^n) < \frac{1}{2^n} < \epsilon\\
&2^n > \frac{1}{\epsilon}\\
&n > \log_2{\frac{1}{\epsilon}}\\
&N = \bigg[\bigg|log_2{\frac{1}{\epsilon}}\bigg|\bigg] + 1
\end{align*}

\end{enumerate}

\end{enumerate}

\end{document}