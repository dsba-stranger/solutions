\documentclass{article}
\usepackage[utf8]{inputenc}
\usepackage{amsmath, amssymb}
\usepackage{mathtools}
\usepackage{tikz}
\usepackage{pgfplots}

\title{DSBA Calculus HW5}
\author{Kirill Korolev, 203-1}
\date{6th of October, 2020}

\begin{document}
	
\maketitle

\begin{enumerate}
	\item Find the domain and the range for the following functions:
	
	\begin{align*}
	&f(x) = \sqrt{2 + x - x^2}\\
	&D(f) = \{x \in \mathbb{R} \:|\: 2 + x - x^2 \geq 0\}\\
	&x_1, x_2 = \frac{-1 \pm \sqrt{1 - 4 \cdot 2 \cdot (-1)}}{-2} = \frac{-1 \pm 3}{-2} = 2; -1\\
	&-(x + 1)(x - 2) \geq 0\\
	&(x + 1)(x - 2) \leq 0\\
	&x \in [-1; 2]
	\end{align*}
	
\begin{tikzpicture}
\begin{axis}
\addplot[smooth,samples=200,domain=-5:5]{sqrt(2 + x - x^2)};
\end{axis}
\end{tikzpicture}		
	
	\begin{center}
		In $y = 2 + x - x^2$ branches are directed downwards. It reaches maximum at $x_0 = \frac{1}{2} \Rightarrow y_0 = \frac{9}{4} \Rightarrow f(x) \leq \sqrt{\frac{9}{4}} = \frac{3}{2}$\\
		Also square root must be greater or equal than zero, hence range $y \in [0; \frac{3}{2}]$
	\end{center}

	\begin{equation*}
	f(x) = 
	\begin{cases*}
		x + 1, \quad -3 \leq x < -1\\
		2 - x^2, \quad x \geq -1
	\end{cases*}
	\end{equation*}
	
\begin{tikzpicture}
\begin{axis}
\addplot[smooth,samples=200,domain=-3:-1]{x + 1};
\addplot[smooth,samples=200,domain=-1:5]{2 - x^2};
\end{axis}
\end{tikzpicture}	
	
	There is no limitations on functions defined on these intervals except that $y = x + 1$ starts from -3, so domain of f(x) is $[-3; +\infty)$. Linear function is less than 0 on $-3 \leq x < -1$ and monotone increases. Parabola has branches directed downwards, so the maximum will be at $f(0) = 2$. Despite the gap between these functions at $x = -1$, the right branch of parabola tends to minus infinity, so the range is $(-\infty, 2]$

\item Use the $\epsilon-\delta$ definition of limit to prove that $\lim_{x \to 9} \sqrt{x} = 3$. Find such $\delta$ such that $|\sqrt{x} - 3| < 0.01$ whenever $0 < |x - 9| < \delta$

\begin{align*}
\forall \epsilon > 0, \exists \delta > 0: \forall x \quad 0 < |x - 9| < \delta \quad \Rightarrow \quad |\sqrt{x} - 3| < \epsilon\\
|\sqrt{x} - 3| = \frac{|x - 9|}{\sqrt{x} + 3} < \frac{\delta}{\sqrt{x} + 3} < \epsilon\\
\forall x, \sqrt{x} + 3 \geq 3 \Rightarrow \delta < (\sqrt{x} + 3) \cdot \epsilon
\end{align*}
So we can take $\delta = \epsilon$

\begin{align*}
&|\sqrt{x} - 3| < 0.01\\
&3 - 0.01 < \sqrt{x} < 3 + 0.01\\
&9 - 6 \cdot 0.01 + 0.01 ^ 2 < x < 9 + 6 \cdot 0.01 + 0.01 ^ 2
\end{align*}
If we take $\delta = 0.01 ^ 2$, then $9 - 0.01 ^ 2 < x < 9 + 0.01 ^ 2$, which satisfies the upper inequality.

\item Find the following limits:


\begin{align*}
\lim_{x \to 0} \frac{x^2 - 1}{2x^2 - x - 1} = \lim_{x \to 0} \frac{(x - 1)(x + 1)}{(x - 1)(2x + 1)} =  \lim_{x \to 0} \frac{x + 1}{2x + 1} = 1
\end{align*}

\begin{align*}
\lim_{x \to 1} \frac{x^2 - 1}{2x^2 - x - 1} = \lim_{x \to 1} \frac{(x - 1)(x + 1)}{(x - 1)(2x + 1)} =  \lim_{x \to 1} \frac{x + 1}{2x + 1} = \frac{2}{3}
\end{align*}

\begin{align*}
\lim_{x \to \infty} \frac{x^2 - 1}{2x^2 - x - 1} = \lim_{x \to \infty} \frac{(x - 1)(x + 1)}{(x - 1)(2x + 1)} =  \lim_{x \to \infty} \frac{x + 1}{2x + 1} = \lim_{x \to \infty} \frac{1 + \frac{1}{x}}{2 + \frac{1}{x}} = \frac{1}{2}
\end{align*}

\begin{align*}
&\lim_{x \to 0} \frac{(1 + x)(1 + 2x)(1 + 3x) - 1}{x} \quad S_3 = \frac{1 + x + 1 + 3x}{2}3 = 3(1 + 2x)\\
&\lim_{x \to 0} \frac{(1 + x)(1 + 2x)(1 + 3x) - 1}{x} = \lim_{x \to 0} \frac{3(1 + 2x) - 1}{x} = \lim_{x \to 0} \frac{2 + 6x}{x} = 6
\end{align*}

\item Find the following limits:
\begin{align*}
&lim_{x \to \infty} \bigg(\sqrt{x^2 + 3x - 1} - \sqrt{x^2 + 7}\bigg) =\\
&lim_{x \to \infty} \frac{3x - 8}{\sqrt{x^2 + 3x - 1} + \sqrt{x^2 + 7}} =\\
&lim_{x \to \infty} \frac{3 - \frac{8}{x}}{\sqrt{1 + \frac{3}{x} - \frac{1}{x^2}} + \sqrt{1 + \frac{7}{x^2}}} = \frac{3}{2}
\end{align*}

\begin{align*}
&\lim_{x \to 3} \frac{\sqrt{6+x} - x}{\sqrt{28 - x} - 5} =\\
&\lim_{x \to 3} \frac{(6 + x - x^2)(\sqrt{28-x} + 5)}{(28 - x - 25)(\sqrt{6 + x} + x)} =\\
&\lim_{x \to 3} \frac{-(x - 3)(x + 2)(\sqrt{28-x} + 5)}{(3 - x)(\sqrt{6 + x} + x)} =\\
&\lim_{x \to 3} \frac{(x + 2)(\sqrt{28-x} + 5)}{(\sqrt{6 + x} + x)} = \frac{5 \cdot 10}{3 + 3} = \frac{25}{3}
\end{align*}

\item Find the following limits:
\begin{align*}
&lim_{x \to +\infty} \frac{3 \cdot 2^x - 7 \cdot 3^x}{2^{x + 2} + 5 \cdot 3^x} =\\
&lim_{x \to +\infty} \frac{3 \cdot (\frac{2}{3})^x - 7}{4 \cdot (\frac{2}{3})^x + 5} = -\frac{7}{5}
\end{align*}

\begin{align*}
&lim_{x \to -\infty} \frac{3 \cdot 2^x - 7 \cdot 3^x}{2^{x + 2} + 5 \cdot 3^x} =\\
&lim_{x \to -\infty} \frac{3 - 7 \cdot (\frac{3}{2})^x}{4 + 5 \cdot (\frac{3}{2})^x} = \frac{3}{4}
\end{align*}

\end{enumerate}

\end{document}