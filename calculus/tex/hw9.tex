\documentclass{article}
\usepackage[utf8]{inputenc}
\usepackage{amsmath, amssymb}
\usepackage{mathtools}
\usepackage{tikz}
\usepackage{pgfplots}
\usetikzlibrary{positioning}

\title{DSBA Calculus HW9}
\author{Kirill Korolev, 203-1}
\date{10th of November, 2020}

\begin{document}
	
\maketitle

\begin{enumerate}
\item Find vertical and oblique asymptotes of the following functions:

\begin{enumerate}
\item $f(x)=\frac{x+1}{x^2+3x-4}$

\[f(x)=\frac{x+1}{x^2+3x-4}=\frac{x+1}{(x+4)(x-1)}\]
Firstly, let's find vertical asymptotes by checking several potential points $x=-4$ and $x=1$.

\begin{align*}
\lim_{x \to -4+} \frac{x+1}{(x+4)(x-1)}=+\infty\\
\lim_{x \to -4-} \frac{x+1}{(x+4)(x-1)}=-\infty\\
\lim_{x \to 1+} \frac{x+1}{(x+4)(x-1)}=+\infty\\
\lim_{x \to 1-} \frac{x+1}{(x+4)(x-1)}=-\infty\\
\end{align*}
So, these are vertical asymptotes $x=-4$ and $x=1$. Then consider oblique asymptotes.
\begin{align*}
k_{+-}=\lim_{x \to \infty} \frac{x+1}{(x+4)(x-1)x}=0\\
b_{+-}=\lim_{x \to \infty} \frac{x+1}{(x+4)(x-1)}=0\\
\end{align*}
Therefore $y = 0$ is an oblique asymptote.

\item $f(x)=\sqrt{\frac{x^3}{x-2}}$

This function is defined while $x \in (-\infty, 0] \cup [2, +\infty)$. So let's check the margin points $x=0$ and $x=2$ by considering one-sided limits.

\begin{align*}
\lim_{x \to 0-} \sqrt{\frac{x^3}{x-2}}=0\\
\lim_{x \to 2+} \sqrt{\frac{x^3}{x-2}}=+\infty\\
\end{align*}

So, $x=2$ is a vertical asymptote. What about oblique asymptotes?
\begin{align*}
k_{+}&=\lim_{x \to +\infty} \frac{\sqrt{\frac{x^3}{x-2}}}{x}=\lim_{x \to +\infty} \sqrt{\frac{x}{x-2}}=1\\
b_{+}&=\lim_{x \to +\infty} \sqrt{\frac{x^3}{x-2}} - x=\lim_{x \to +\infty} \frac{\frac{x^3}{x-2} - x^2}{\sqrt{\frac{x^3}{x-2}} + x}=\lim_{x \to +\infty} \frac{2x^2}{(x-2)(\sqrt{\frac{x^3}{x-2}} + x)}=\\
&=\lim_{x \to +\infty} \frac{2x^2}{(x^2-2x)(\sqrt{\frac{x}{x-2}} + 1)}=1\\
k_{-}&=\lim_{x \to -\infty} \frac{\sqrt{\frac{x^3}{x-2}}}{x}=\lim_{x \to -\infty} -\sqrt{\frac{x}{x-2}}=-1\\
b_{-}&=\lim_{x \to -\infty} \sqrt{\frac{x^3}{x-2}} + x=\lim_{x \to -\infty} \frac{\frac{x^3}{x-2} + x^2}{\sqrt{\frac{x^3}{x-2}} + x}=\lim_{x \to -\infty} \frac{-2x^2}{(x-2)(\sqrt{\frac{x^3}{x-2}} + x)}=\\
=&\lim_{x \to -\infty} \frac{-2x^2}{(x^2-2x)(\sqrt{\frac{x}{x-2}} + 1)}=-1
\end{align*}

$y=x+1$ and $y=-x-1$ are oblique asymptotes.

\item $f(x)=\sqrt{x^2-1}-x$

Function is defined while $x \in (-\infty, -1] \cup [1, +\infty)$
\begin{align*}
\lim_{x \to -1-} \sqrt{x^2-1}-x=1\\
\lim_{x \to 1+} \sqrt{x^2-1}-x=-1\\
\end{align*}

These limits are finite, therefore there are no vertical asymptotes.

\begin{align*}
k_{+-}&=\lim_{x \to \infty} \frac{\sqrt{x^2-1}-x}{x}=\lim_{x \to \infty} \frac{|x|\sqrt{1-\frac{1}{x^2}}-x}{x}=\\
&=\lim_{x \to \infty} sgn(x)\sqrt{1-\frac{1}{x^2}}-1=sgn(x)-1 = 0; -2\\
b_{+}&=\lim_{x \to +\infty} \sqrt{x^2-1}-x=\lim_{x \to +\infty} \frac{-1}{\sqrt{x^2-1}+x}=0\\
b_{-}&=\lim_{x \to -\infty} \sqrt{x^2-1}-x+2x=\lim_{x \to -\infty} \sqrt{x^2-1}+x=\lim_{x \to -\infty} \frac{-1}{\sqrt{x^2-1}-x}=0
\end{align*}

Oblique asymptotes are $y=-2x$ and $y=0$.

\item $f(x)=\frac{\sqrt{4x^4+1}}{|x|}$

The only suspicious point is $x=0$.

\begin{align*}
\lim_{x \to 0-} \frac{\sqrt{4x^4+1}}{|x|}=+\infty\\
\lim_{x \to 0+} \frac{\sqrt{4x^4+1}}{|x|}=+\infty\\
\end{align*}

$x=0$ is a vertical asymptote.

\begin{align*}
k_{+}&=\lim_{x \to +\infty} \frac{\sqrt{4x^4+1}}{x^2}=\lim_{x \to +\infty} \frac{x^2\sqrt{4+\frac{1}{x^4}}}{x^2}=2\\
k_{-}&=\lim_{x \to -\infty} \frac{\sqrt{4x^4+1}}{-x^2}=\lim_{x \to -\infty} \frac{x^2\sqrt{4+\frac{1}{x^4}}}{-x^2}=-2\\
b_{+}&=\lim_{x \to +\infty} \frac{\sqrt{4x^4+1}}{x}-2x=\lim_{x \to +\infty} \frac{\sqrt{4x^4+1}-2x^2}{x}=\lim_{x \to +\infty} \frac{1}{x(\sqrt{4x^4+1}+2x^2)}=0\\
b_{-}&=\lim_{x \to -\infty} \frac{\sqrt{4x^4+1}}{x}+2x=\lim_{x \to -\infty} \frac{\sqrt{4x^4+1}+2x^2}{x}=\lim_{x \to -\infty} \frac{1}{x(\sqrt{4x^4+1}-2x^2)}=0
\end{align*}

$y=2x$ and $y=-2x$ are oblique asymptotes.

\item $f(x)=x\arctan{\frac{x}{2}}$

There are no vertical asymptotes, function is defined everywhere.

\begin{align*}
k_{+}&=\lim_{x \to +\infty} \frac{x\arctan{\frac{x}{2}}}{x}=\arctan{\frac{x}{2}}=\frac{\pi}{2}\\
k_{-}&=\lim_{x \to -\infty} \frac{x\arctan{\frac{x}{2}}}{x}=\arctan{\frac{x}{2}}=-\frac{\pi}{2}\\
\end{align*}

I'll use the following identity $\arctan{x}+\arctan{\frac{1}{x}}=\frac{\pi}{2}, \: x > 0$ to find the next limit. Brief proof of it, for example, if we've got a right-triangle with sides $1$ and $x$, then $\tan{\alpha}=x$ and $\tan{\beta}=\frac{1}{x}$, we know that the sum of angles are equal to $\frac{\pi}{2}$, so then $\alpha+\beta=\arctan{x}+\arctan{\frac{1}{x}}=\frac{\pi}{2}$. For negative $x$ we know that $\arctan{x}$ is an odd function, so $\arctan{(-x)}=-\arctan{x}$, therefore $\arctan{(-x)}+\arctan{(-\frac{1}{x})}=-(\arctan{x}+\arctan{\frac{1}{x}})=-\frac{\pi}{2}$.

Also, I'll use the equivalence $\arctan{x} \sim x, \: x \to 0$.
\begin{align*}
b_{+}&=\lim_{x \to +\infty} x\arctan{\frac{x}{2}}-\frac{\pi}{2}x=\bigg|\bigg|t=\frac{x}{2}\bigg|\bigg|=\lim_{t \to +\infty} 2t(\arctan{t}-\frac{\pi}{2})=\lim_{t \to +\infty} -2t\arctan{\frac{1}{t}}=\\
&=\lim_{t \to +\infty} -\frac{2t}{t}=-2\\
b_{-}&=\lim_{x \to -\infty} x\arctan{\frac{x}{2}}+\frac{\pi}{2}x=\bigg|\bigg|t=\frac{x}{2}\bigg|\bigg|=\lim_{t \to -\infty} 2t(\arctan{t}+\frac{\pi}{2})=\lim_{t \to -\infty} -2t\arctan{\frac{1}{t}}=\\
&=\lim_{t \to -\infty} -\frac{2t}{t}=-2
\end{align*}

Therefore, we got $y=\frac{\pi}{2}x-2$ and $y=-\frac{\pi}{2}x-2$.

\end{enumerate}

\item Use the definition to find the derivatives of the following functions:

\begin{enumerate}
\item $f(x) = 8x^2 - x + 2$

\begin{align*}
f^\prime(x)&=\lim_{h \to 0} \frac{8(x+h)^2-(x+h)+2-(8x^2-x+2)}{h}=\\
&=\lim_{h \to 0} \frac{8x^2+16xh+8h^2-x-h+2-8x^2+x-2}{h}=\\
&=\lim_{h \to 0} \frac{16xh+8h^2-h}{h}=\lim_{h \to 0} (16x+8h-1)=16x-1
\end{align*}

\item $f(x) = 2\sqrt{x+4} \quad x_0=5$

\begin{align*}
f^\prime(x_0)&=\lim_{h \to 0} \frac{2\sqrt{x_0+h+4}-2\sqrt{x_0+4}}{h}=\lim_{h \to 0} \frac{2\sqrt{h+9}-6}{h}=\\
&=\lim_{h \to 0} \frac{2(h+9-9)}{h(\sqrt{h+9}+3)}=\frac{2}{6}=\frac{1}{3}
\end{align*}

\item $f(x)=\cos{x}$

\begin{align*}
f^\prime(x)&=\lim_{h \to 0} \frac{\cos{(x+h)}-\cos{x}}{h}=\lim_{h \to 0} \underbrace{\frac{-2\sin{\frac{h}{2}}\sin{\frac{2x+h}{2}}}{h}}_{\text{apply first remarkable limit}}=\lim_{h \to 0} -\sin{\frac{2x+h}{2}}=-\sin{x}
\end{align*}

\end{enumerate}

\item Compute the derivatives of the following functions using basic differentiation rules:
\begin{enumerate}
\item $f(x)=5^{\cos{x}} \cdot \ln{x} + \frac{x^2+\sin{x}}{\sqrt{5x^2+3x-7}}$

\begin{align*}
f^\prime(x)=5^{\cos{x}} \cdot \ln{5} \cdot (-\sin{x}) \cdot \ln{x} + \frac{5^{\cos{x}}}{x} +\\
+\frac{(2x+\cos{x})\sqrt{5x^2+3x-7}-(x^2+\sin{x})\frac{10x+3}{2\sqrt{5x^2+3x-7}}}{5x^2+3x-7}
\end{align*}

\item $f(x)=(x^2+3)\cdot\tan{\sqrt{x}}+\frac{5^x}{7x-\ln{x}}$

\begin{align*}
f^\prime(x)=2x \cdot \tan{\sqrt{x}} + \frac{x^2+3}{2\sqrt{x}\cos^2{\sqrt{x}}} + \frac{5^x \cdot \ln{5} \cdot (7x-\ln{x}) - 5^x(7 - \frac{1}{x})}{(7x-\ln{x})^2}
\end{align*}

\end{enumerate}

\item Compute the derivatives of the following functions:

\begin{enumerate}
\item $f(x)=(\arctan{x})^{\cos^2{x}}$

\begin{align*}
f^\prime(x)=(e^{\ln{(\arctan{x})^{\cos^2{x}}}})^\prime=(e^{\cos^2{(x)}\ln{(\arctan{x})}})^\prime=(\arctan{x})^{\cos^2{x}}(-2\cos{x}\sin{x}\ln{(\arctan{x})} +\\
+\frac{\cos^2{(x)}}{\arctan{x}\cdot(1+x^2)})
\end{align*}

\item $f(x)=\frac{e^{\arccos{x}}(x+7)^9}{(1+x^2)^4}=e^{\arccos{x}}\frac{(x+7)^9}{(1+x^2)^4}$

\begin{align*}
f^\prime(x)=-\frac{e^{\arccos{x}}\frac{(x+7)^9}{(1+x^2)^4}}{\sqrt{1-x^2}}+e^{\arccos{x}}\frac{9(x+7)^8(1+x^2)^4-4(x+7)^9(1+x^2)^3\cdot2x}{(1+x^2)^8}
\end{align*}

\end{enumerate}

\end{enumerate}

\end{document}