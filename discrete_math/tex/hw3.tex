\documentclass{article}
\usepackage{amsmath}
\usepackage[utf8]{inputenc}

\title{Discrete Mathematics HW3}
\author{Kirill Korolev, DSBA 203-1}
\date{30th of September 2020}

\begin{document}

\maketitle

\begin{enumerate}
\setcounter{enumi}{5}

\item Suppose you are given with the following question when taking a multiple choice test:

What percentage of answers to this question are correct?

\begin{enumerate}
\item 50\%
\item 25\%
\item 0\%
\item 50\%
\end{enumerate}

Choosing (a) and (d) is correct because 2 / 4 answers are exactly 50\%. Also, choosing (b) will be correct, because 1 / 4 is 25\% of answers.

\item Prove formally that $init(rev(init([student]))) = [nedut]$.

\textbf{By definition of $init(s), \forall s \in S(A)$}

\begin{align*}
init([student]) = s : init([tudent]) = s : (t : init([udent])) = ... = s : (t : (u : (d : (e : (n : [])))))
\end{align*}

\textbf{By definition of $rev(s), \forall s \in S(A)$}

\begin{align*}
rev(init(student)) = rev(s : (t : (u : (d : (e : (n : [])))))) =\\
app(rev(t : (u : (d : (e : (n : []))))), [s]) =\\ app(app(rev(u : (d : (e : (n : [])))), [t]), [s]) = ... = \\ = app(app(app(app(app(app([], [n]), [e]), [d]), [u]), [t]), [s]) 
\end{align*}

\textbf{Roll it down by definition of $app(s, t), \forall s, t \in S(A)$}
\begin{align*}
app(app(app(app(app(app([], [n]), [e]), [d]), [u]), [t]), [s]) =\\
app(app(app(app(app([n], [e]), [d]), [u]), [t]), [s]) = \\
app(app(app(app(n: app([], [e]), [d]), [u]), [t]), [s]) = \\
app(app(app(app(n: [e], [d]), [u]), [t]), [s]) = ... = \\
= n : (e : (d : (u : (t : (s : [])))))
\end{align*}

\textbf{Finally, after applying $init$}
\begin{align*}
init(n : (e : (d : (u : (t : (s : [])))))) =\\ n : init(e : (d : (u : (t : (s : []))))) = ... =\\ n : (e : (d : (u : (t : init([s]))))) =\\ n : (e : (d : (u : (t : [])))) = [nedut]
\end{align*}

\item Prove formally that for every $s, t \in S(A)$ the following statements hold:

\begin{enumerate}
\item $lh(s) = 0 \iff s = []$

If $s = []$, then by definition of length function $lh([]) = 0$.

If $lh(s) = 0$, suppose $s = x : s ^ \prime$. Then by definition $lh(x : s ^ \prime) = 1 + lh(s ^ \prime) \neq 0$, which contradicts with former condition. Hence $s = []$.

\item $lh(app(s, t)) = lh(s) + lh(t)$

Proof by induction.

\paragraph{Base $s = []$:}
$lh(app([], t)) = lh(t) = lh([]) + lh(t) = lh(s) + lh(t)$

\paragraph{Inductive step:}
$lh(app(x : s, t)) = lh(x : app(s, t)) = 1 + lh(app(s, t)) = 1 + lh(s) + lh(t) = lh(x : s) + lh(t)$

\item $lh(rev(s)) = lh(s)$

To prove this let's use statement (b). Proof by induction.

\paragraph{Base $s = []$:}
$lh(rev([])) = lh([]) = 0 = lh([]) = lh(s)$

\paragraph{Inductive step:}
$lh(rev(x : s)) = lh(app(rev(s), [x])) = lh(rev(s)) + lh([x]) = lh(s) + lh([x]) = lh(s) + 1 + lh([]) = lh(s) + 1 = lh(x : s)$

\end{enumerate}

\item Prove that $app(t, s) = app(r, s)$ implies $t = r$ for every $s, t, r \in S(A)$:

In proof I'm going to use lemmas proved in seminars.

\[app(t, s) = rev(rev(app(t, s)) = rev(app(rev(s), rev(t)))\]
\[app(r, s) = rev(rev(app(r, s)) = rev(app(rev(s), rev(r)))\]

The final step in order to use lemma, which implies the same as the statement which is being proved but for second argument, is to prove that $rev(s) = rev(t)$ implies $s = t$.

Let's prove this additional statement by induction.

\paragraph{Base:} $rev(s) = rev([]) = [] = rev(t) = []$
\paragraph{Inductive step:} $rev(x : s) = $ 

\end{enumerate}

\end{document}
