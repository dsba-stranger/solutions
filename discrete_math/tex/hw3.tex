\documentclass{article}
\usepackage{amsmath}
\usepackage[utf8]{inputenc}

\title{Discrete Mathematics HW3}
\author{Kirill Korolev, DSBA 203-1}
\date{30th of September 2020}

\begin{document}

\maketitle

\begin{enumerate}
\setcounter{enumi}{5}

\item Suppose you are given with the following question when taking a multiple choice test:

What percentage of answers to this question are correct?

\begin{enumerate}
\item 50\%
\item 25\%
\item 0\%
\item 50\%
\end{enumerate}

Choosing (a) and (d) is correct because 2 / 4 answers are exactly 50\%. Also, choosing (b) will be correct, because 1 / 4 is 25\% of answers.

\item Prove formally that $init(rev(init([student]))) = [nedut]$.

\textbf{By definition of $init(s), \forall s \in S(A)$}

\begin{align*}
init([student]) = s : init([tudent]) = s : (t : init([udent])) = ... = s : (t : (u : (d : (e : (n : [])))))
\end{align*}

\textbf{By definition of $rev(s), \forall s \in S(A)$}

\begin{align*}
rev(init(student)) = rev(s : (t : (u : (d : (e : (n : [])))))) =\\
app(rev(t : (u : (d : (e : (n : []))))), [s]) =\\ app(app(rev(u : (d : (e : (n : [])))), [t]), [s]) = ... = \\ = app(app(app(app(app(app([], [n]), [e]), [d]), [u]), [t]), [s]) 
\end{align*}

\textbf{Roll it down by definition of $app(s, t), \forall s, t \in S(A)$}
\begin{align*}
app(app(app(app(app(app([], [n]), [e]), [d]), [u]), [t]), [s]) =\\
app(app(app(app(app([n], [e]), [d]), [u]), [t]), [s]) = \\
app(app(app(app(n: app([], [e]), [d]), [u]), [t]), [s]) = \\
app(app(app(app(n: [e], [d]), [u]), [t]), [s]) = ... = \\
= n : (e : (d : (u : (t : (s : [])))))
\end{align*}

\textbf{Finally, after applying $init$}
\begin{align*}
init(n : (e : (d : (u : (t : (s : [])))))) =\\ n : init(e : (d : (u : (t : (s : []))))) = ... =\\ n : (e : (d : (u : (t : init([s]))))) =\\ n : (e : (d : (u : (t : [])))) = [nedut]
\end{align*}

\end{enumerate}

\end{document}
