\documentclass{article}
\usepackage{amsmath, amssymb}
\usepackage{mathtools}

\title{DSBA Discrete Mathematics HW4}
\author{Kirill Korolev 203-1}
\date{12th of October, 2020}

\begin{document}
\maketitle

\begin{enumerate}
\item Prove that for each natural $n > 1$,
\[\sum_{k=1}^{n} k^2 = \frac{n(n+1)(2n+1)}{6}\]

\textbf{Proof by induction.}

\begin{align*}
IB.& \quad 1^2 = \frac{1 \cdot 2 \cdot 3}{6} = 1\\
IS.& \quad \sum_{k=1}^{n + 1} k^2 = \underbrace{\frac{n(n+1)(2n+1)}{6}}_{ind. hyp.} + (n+1)^2=\\
&=\frac{(n+1)\bigg(n(2n+1) + 6(n+1)\bigg)}{6} = \frac{(n+1)(2n^2 + n + 6n + 6)}{6}\\
&n_1, n_2 = \frac{-7 \pm \sqrt{49 - 4 \cdot 6 \cdot 2}}{4} = \frac{-7 \pm 1}{4} = -\frac{3}{2}; -2\\
&\frac{(n+1)(2n^2 + 7n + 6)}{6} = \frac{(n+1)(n+2)(2n+3)}{6}
\end{align*}

\item In some country, there are finitely many cities, each two of which are connected by a one-way road. Prove that there is a city from which any other is accessible.

Let $C_n$ will be a set of $n$ cities. Denote $P(x, y)$ as path from $x$ to $y$ and $R(x, y)$ as one-way road from $x$ to $y$, where $x, y \in C_n$. 

$P(x, y) = \exists \{z_i\} : R(x, z_1) \land R(z_1, z_2) \land ... \land R(z_m, y) \quad z_i \in C_n, \: m \in \mathbb{N}$.

Then we need to prove $A(n) = \exists x \in C_n : \forall y \in C_n \: P(x, y)$

Also, by condition holds $\forall x, y \in C_n \quad R(x, y) \lor R(y, x)$


\textbf{Proof by induction on $A(n)$.}
$A(2)$ is obvious by condition. Suppose that statement is valid for $A(n)$. So there is a city $x_0$ with path to other $n - 1$ cities:
\[\exists x_0 \in C_n : \forall y \in C_n \quad P(x_0, y) \]

Let's add another city $y_0$. Then by condition $y_0$ is also connected with other $n$ cities. There are two cases.

\paragraph{Case 1}
$\exists z \in C_n \: (P(x_0, z) \land R(z, y_0)) \to P(x_0, y_0) \Rightarrow x_0 $ is the source city. 
\paragraph{Case 2}
$\forall z \in C_n \: R(y_0, z) \Rightarrow y_0 $ is the source city.

Therefore, the statement has been proved.

\item Prove that for each natural $n$, there exists some $k$ s.t. $1 + \frac{1}{2} + ... + \frac{1}{k} \geq n$

\paragraph{Proof by induction.}
\begin{align*}
&IB. \quad 1 \geq 1\\
&\Delta_{k,m} = \frac{1}{k+1} + ... + \frac{1}{k+m} \quad k, m \in \mathbb{N}\\
&IS. \quad 1 + \frac{1}{2} + ... + \frac{1}{k} + \Delta_{k,m} \geq n + \Delta_{k,m}
\end{align*}

To prove given statement we need to show that $n + \Delta_{k,m} \geq n + 1 \Rightarrow \Delta_{k,m} \geq 1$. If we take $m = k$ then:
\[\frac{1}{k+1} + ... + \frac{1}{2k} \geq 2k \cdot \frac{1}{2k} = 1\]

\end{enumerate}

\end{document}