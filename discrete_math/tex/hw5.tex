\documentclass{article}
\usepackage{amsmath, amssymb}
\usepackage{mathtools}
\usepackage[legalpaper,margin=1in]{geometry}

\title{DSBA Discrete Mathematics HW5}
\author{Kirill Korolev 203-1}
\date{21th of October, 2020}

\begin{document}
\maketitle

\begin{enumerate}
\item Find the last two digits of the number $99^{1000}$ in decimal notation.

Let's use corollary from Fermat's Little Theorem that if $gcd(a,m)=1$ and $x \equiv y\:(\varphi(m))$ then $a^x \equiv a^y \:(m)$

\begin{align*}
gcd(99,1000) = 1\\
\varphi(100) = \varphi(5^2 \cdot 2^2) = 100(1 - \frac{1}{5})(1 - \frac{1}{2}) = 40 \Rightarrow\\
1000 \equiv 0 \: (40) \Rightarrow 99^{1000} \equiv (99)^0 \equiv 1 \: (100)
\end{align*}

\item Prove that $a^3$ and $b^3$ result in the same remainder when divided by $a - b$.

Same remainder is equivalent to $a^3 \equiv b^3 \: (a - b)$ or $(a - b) \mid (a^3 - b^3)$. That's true because $a^3 - b^3 = (a - b)(a^2 + ab + b^2)$.

\item If $11 \mid (5m + 3n)$, then $11 \mid (9m + n)$

We need to show that if $5m \equiv -3n \: (11)$ then $9m \equiv -n \: (11)$. By the properties of congruences we need another assumption that $4m \equiv 2n \: (11)$ or because $gcd(2, 11) = 1$ we can divide by 2, then $2m \equiv n \: (11)$.

\begin{align*}
gcd(3,11) = 1 \Rightarrow\\
5m \equiv -3n \: (11) \Rightarrow -6m \equiv -3n \: (11) \Rightarrow 2m \equiv n \: (11)
\end{align*}

That's exactly what we needed, therefore, the proof is over.

\item In a certain programming language, there is a type $Int$ housing all integers from the range $[-M; M - 1]$,
where M is a large positive integer. If an integer x is out of that range (that is, there is an overflow), then $x$
is automatically presented in $Int$ as some other integer $I(x)$ within the range. On overflow, the value wraps
around so that

\[I(x) = remainder(x+M,2M)-M\]

for any integer $x$. Prove that for every integers $x$ and $y$, the following hold:

\begin{enumerate}
\item $I(x)=I(I(x))$

\begin{align*}
&I(I(x)) = remainder(remainder(x+M,2M)-M+M,2M)-M =\\
&=remainder(remainder(x+M,2M),2M)-M = remainder(x+M,2M) - M = I(x)
\end{align*}

Because $remainder(remainder(x, y), y) = remainder(x, y) \quad \forall x, y \in \mathbb{Z}$ as $remainder(x, y) < y$

\item $I(x+y)=I(I(x)+I(y))$

\begin{align*}
&I(I(x)+I(y))=remainder(remainder(x+M,2M)-M+remainder(y+M,2M)-M+M,2M)-M=\\
&=remainder(remainder(x+M,2M)+remainder(y+M,2M)-M,2M)-M=\\
&=remainder(\underbrace{remainder(x+y+2M,2M)}_{\text{by sum of congruences}}-M,2M)-M=\\
&=remainder(\underbrace{remainder(x+y,2M)}_{2M \equiv 0\:(2M)}-M,2M)-M=\\
&=remainder(x+y-M,2M)-M=remainder(x+y+M,2M)-M=I(x+y)
\end{align*}
//TODO: Add explanations

\item $I(xy)=I(I(x) \cdot I(y))$

\begin{align*}
I(I(x) \cdot I(y)) = remainder((remainder(x+M,2M)-M)(remainder(y+M,2M)-M)+M,2M)-M=\\
\end{align*}
//TODO: Finish

\end{enumerate}

\item Suppose a number $a > 1$ is divisible by 2 but not by 4. Then $a$ has as many positive even divisors as
it has positive odd divisors.

By fundamental theorem of arithmetic we can factorize $a$ on product of prime numbers. Condition that $a$ is divisible by 2 but not by 4 means that 2 occurs in factorization in a first power, otherwise we'd divide by $2^2$.

\[a=2 \cdot 3^{\alpha_1} \cdot 5^{\alpha_2} \cdot ... \cdot p^{\alpha_k}\]

All even divisors are obtained by taking 2 and some combination of $3, 5, \ldots{}, p$ in various powers. For example, $2, 2 \cdot 3, 2 \cdot 3^2, 2 \cdot 5, \ldots{}$ or even the whole number as it is divisible by 2, in other words, is even. Let's count the number of combinations for even divisors.

\begin{align*}
m = \alpha_1 + \alpha_2 + \ldots{} + \alpha_k\\
C^0_m+C^1_m+\ldots{}+C^m_m = 2^m
\end{align*} 

For odd divisors the sum is the same except there is no summand $C^0_m$, but $C^0_m=1$ and we don't need to forget adding 1 as odd divisor, so the result is the same: $1 + C^1_m+\ldots{}+C^m_m=2^m$. Therefore, there are as many even divisors as odd ones.

\end{enumerate}

\end{document}