\documentclass{article}
\usepackage{amsmath, amssymb}
\usepackage{mathtools}

\title{DSBA Linear Algebra HW4}
\author{Kirill Korolev 203-1}
\date{10th of October, 2020}

\makeatletter
\renewcommand*\env@matrix[1][*\c@MaxMatrixCols c]{%
  \hskip -\arraycolsep
  \let\@ifnextchar\new@ifnextchar
  \array{#1}}
\makeatother

\begin{document}
\maketitle

\begin{enumerate}

\item Solve the following system:

\begin{align*}
&\begin{bmatrix}[ccccc|c]
6 & 4 & 5 & 2 & 3 & 1\\
9 & 6 & 1 & 3 & 2 & 2\\
3 & 2 & -2 & 1 & 0 & -7\\
3 & 2 & 4 & 1 & 2 & 3
\end{bmatrix}
\to
\begin{bmatrix}[ccccc|c]
3 & 2 & 4 & 1 & 2 & 3\\
6 & 4 & 5 & 2 & 3 & 1\\
9 & 6 & 1 & 3 & 2 & 2\\
3 & 2 & -2 & 1 & 0 & -7\\
\end{bmatrix}
\to
\begin{bmatrix}[ccccc|c]
3 & 2 & 4 & 1 & 2 & 3\\
0 & 0 & -3 & 0 & -1 & -5\\
0 & 0 & -11 & 0 & -4 & -7\\
0 & 0 & -6 & 0 & -2 & -10
\end{bmatrix}
\to\\
&\begin{bmatrix}[ccccc|c]
3 & 2 & 4 & 1 & 2 & 3\\
0 & 0 & -3 & 0 & -1 & -5\\
0 & 0 & -11 & 0 & -4 & -7\\
0 & 0 & 0 & 0 & 0 & 0
\end{bmatrix}
\to
\begin{bmatrix}[ccccc|c]
3 & 2 & 4 & 1 & 2 & 3\\
0 & 0 & -3 & 0 & -1 & -5\\
0 & 0 & 0 & 0 & -\frac{1}{3} & \frac{34}{3}\\
0 & 0 & 0 & 0 & 0 & 0
\end{bmatrix}\\
&\begin{cases}
x_5 = -34\\
x_3 = \frac{5}{3} + \frac{34}{3} = 13\\
x_1 = 1 + 34 \cdot \frac{2}{3} - \frac{x_4}{3} - 13 \cdot \frac{4}{3} - \frac{2x_2}{3} = \frac{19}{3} - \frac{2x_2}{3} - \frac{x_4}{3}
\end{cases}\\
&X = \begin{bmatrix}
\frac{19}{3} - \frac{2c_1}{3} - \frac{c_2}{3}\\c_1\\13\\c_2\\-34
\end{bmatrix} \quad \forall c_1, c_2 \in \mathbb{R}
\end{align*}

\newpage
\item Solve the following system with a parameter:

\begin{align*}
&\begin{bmatrix}[cccc|c]
\lambda & 1 & 1 & 1 & 1\\
1 & \lambda & 1 & 1 & 1\\
1 & 1 & \lambda & 1 & 1\\
1 & 1 & 1 & \lambda & 1\\
\end{bmatrix}
\to
\begin{bmatrix}[cccc|c]
1 & 1 & 1 & \lambda & 1\\
1 & 1 & \lambda & 1 & 1\\
1 & \lambda & 1 & 1 & 1\\
\lambda & 1 & 1 & 1 & 1\\
\end{bmatrix}
\to
\begin{bmatrix}[cccc|c]
1 & 1 & 1 & \lambda & 1\\
0 & 0 & \lambda - 1 & 1 - \lambda & 0\\
0 & \lambda - 1 & 0 & 1 -\lambda  & 0\\
0 & 1 - \lambda & 1 - \lambda & 1 -\lambda^2 & 1-\lambda\\
\end{bmatrix}
\to\\
&\begin{bmatrix}[cccc|c]
1 & 1 & 1 & \lambda & 1\\
0 & 0 & \lambda - 1 & 1 - \lambda & 0\\
0 & \lambda - 1 & 0 & 1 -\lambda  & 0\\
0 & 0 & 1 - \lambda & 2 - \lambda - \lambda^2 & 1-\lambda\\
\end{bmatrix}
\to
\begin{bmatrix}[cccc|c]
1 & 1 & 1 & \lambda & 1\\
0 & 1 & 0 & -1 & 0\\
0 & 0 & 1 & -1 & 0\\
0 & 0 & 1 - \lambda & 2 - \lambda - \lambda^2 & 1-\lambda\\
\end{bmatrix}
\to\\
&\begin{bmatrix}[cccc|c]
1 & 1 & 1 & \lambda & 1\\
0 & 1 & 0 & -1 & 0\\
0 & 0 & 1 & -1 & 0\\
0 & 0 & 0 & 3 - 2\lambda - \lambda^2 & 1-\lambda\\
\end{bmatrix}\\
\end{align*}

If $(\lambda + 3)(\lambda - 1) \neq 0$:
\begin{align*}
&\begin{cases}
x_1 + x_2 + x_3 + \lambda x_4 = 1\\
x_2 - x_4 = 0\\
x_3 - x_4 = 0\\
-(\lambda + 3)(\lambda - 1) x_4 = 1-\lambda
\end{cases}
\iff
\begin{cases}
x_1 = 1 - \frac{2 + \lambda}{\lambda + 3} = \frac{1}{\lambda + 3}\\
x_2 = x_3 = x_4\\
x_4 = \frac{1}{\lambda + 3}
\end{cases}\\
&X = \begin{bmatrix}
\frac{1}{\lambda + 3}\\\frac{1}{\lambda + 3}\\\frac{1}{\lambda + 3}\\\frac{1}{\lambda + 3}
\end{bmatrix}
\end{align*}
If $\lambda = 1$:
\begin{align*}
&\begin{cases}
x_1 = 1 - 3c\\
x_2 = x_3 = x_4 = c \quad c \in \mathbb{R}
\end{cases}\\
&X = \begin{bmatrix}
1 - 3c\\c\\c\\c
\end{bmatrix} \quad \forall c \in \mathbb{R}
\end{align*}
If $\lambda = -3$ there are no solutions.

\newpage
\item Find a polynomial $f(x) = ax^3 + bx^2 + cx + d$ such that

\[f(-2) = -1, f(-1) = 2, f(1) = 14, f(2) = 35\]

We can rewrite it as system of equations with unknowns $a, b, c, d$:
\begin{align*}
&\begin{bmatrix}[cccc|c]
-8 & 4 & -2 & \lambda & -1\\
-1 & 1 & -1 & \lambda & 2\\
1 & 1 & 1 & \lambda & 14\\
8 & 4 & 2 & \lambda & 35
\end{bmatrix}
\to
\begin{bmatrix}[cccc|c]
1 & 1 & 1 & \lambda & 14\\
-8 & 4 & -2 & \lambda & -1\\
-1 & 1 & -1 & \lambda & 2\\
8 & 4 & 2 & \lambda & 35
\end{bmatrix}
\to
\begin{bmatrix}[cccc|c]
1 & 1 & 1 & \lambda & 14\\
0 & 8 & 0 & 2\lambda & 34\\
0 & 2 & 0 & 2\lambda & 16\\
8 & 4 & 2 & \lambda & 35
\end{bmatrix}
\to\\
&\begin{bmatrix}[cccc|c]
1 & 1 & 1 & \lambda & 14\\
0 & 1 & 0 & \lambda & 8\\
0 & 4 & 0 & \lambda & 17\\
0 & -4 & -6 & -7\lambda & -77
\end{bmatrix}
\to
\begin{bmatrix}[cccc|c]
1 & 1 & 1 & \lambda & 14\\
0 & 1 & 0 & \lambda & 8\\
0 & 4 & 0 & \lambda & 17\\
0 & 0 & -6 & -6\lambda & -60
\end{bmatrix}
\to
\begin{bmatrix}[cccc|c]
1 & 1 & 1 & \lambda & 14\\
0 & 1 & 0 & \lambda & 8\\
0 & 0 & 0 & -3\lambda & -15\\
0 & 0 & -6 & -6\lambda & -60
\end{bmatrix}
\to\\
&\begin{bmatrix}[cccc|c]
1 & 1 & 1 & \lambda & 14\\
0 & 1 & 0 & \lambda & 8\\
0 & 0 & 1 & \lambda & 10\\
0 & 0 & 0 & \lambda & 5\\
\end{bmatrix}
\end{align*}

\begin{align*}
\lambda = 5 \Rightarrow d = 5\\
c + 5 = 10 \Rightarrow c = 5\\
b + 5 = 8 \Rightarrow b = 3\\
a + 3 + 5 + 5 = 14 \Rightarrow a = 1\\
\end{align*}
Hence the polynomial will be in a form of:
\[f(x) = x^3 + 3x^2 + 5x + 5\]



\item Find numbers $a, b, c \in \mathbb{R}$ such that the following equality holds true:

\begin{align*}
\frac{x(5+x)}{(1-x)(2+x^2)} = \frac{a}{1-x} + \frac{b + cx}{2+x^2}\\
x(5+x) = a(2+x^2) + (b + cx)(1 - x)\\
x^2 + 5x = (a - c)x^2 + (c - b)x + 2a + b
\end{align*}
Coefficients of members with same powers should be equal, therefore:

\begin{align*}
\begin{cases}
a - c = 1\\
c - b = 5\\
2a + b = 0
\end{cases} 
\iff
\begin{cases}
-\frac{b}{2} - b - 5 = 1\\
c = b + 5\\
a = -\frac{b}{2}
\end{cases}
\iff
\begin{cases}
b = -4\\
c = 1\\
a = 2
\end{cases}
\end{align*}

Indeed $2(2 + x^2) + (x - 4)(1 - x) = 4 + 2x^2 + x - x^2 - 4 + 4x = x^2 + 5x = x(5+x)$

\end{enumerate}

\end{document}