\documentclass{article}
\usepackage{amsmath, amssymb}
\usepackage{mathtools}

\title{DSBA Linear Algebra HW5}
\author{Kirill Korolev 203-1}
\date{17th of October, 2020}

\makeatletter
\renewcommand*\env@matrix[1][*\c@MaxMatrixCols c]{%
  \hskip -\arraycolsep
  \let\@ifnextchar\new@ifnextchar
  \array{#1}}
\makeatother

\begin{document}
\maketitle

\begin{enumerate}

\item Find the number of inversions in the following ordered line:

\[(9,1,3,7,2,8,5,4,6)\]

Inversion is such pair $(f(i), f(j))$ of permutation $f$ for which valid $i < j$ and $f(i) > f(j)$. Basically, for one-line notation number of inversions means number of non-sorted pairs in ascending order.

Let's list these pairs for each number:

\begin{description}
\item 9: 1, 3, 7, 2, 8, 5, 4, 6
\item 1: none
\item 3: 2
\item 7: 2, 5, 4, 6
\item 2: none
\item 8: 5, 4, 6
\item 5: 4
\item 4: none
\item 6: none
\end{description}

Overall there are $8 + 1 + 4 + 3 + 1 = 17$ inversions.

\item Let $f = \begin{pmatrix}2&3&5&1&4\\4&1&2&3&5\end{pmatrix}$ and $g = \begin{pmatrix}1&5&3&4&2\\5&4&1&2&3\end{pmatrix}$

\begin{enumerate}
\item cycle notation for $f$ and $g$

Each permutation can be represented as product of independent cycles whereas a cycle is a list of $(a_1, a_2, ..., a_k)$ with mapping $a_1 \to a_2, a_2 \to a_3, ..., a_k \to a_1$.

$f = (245)(13)$ and $g = (15423)$

\item $f \circ g$ (using two-line notation)

\[
\begin{pmatrix}
1&5&3&4&2\\
5&4&1&2&3\\
\hline
5&4&1&2&3\\
2&5&3&4&1
\end{pmatrix}
\Rightarrow f \circ g = 
\begin{pmatrix}
1&5&3&4&2\\
2&5&3&4&1
\end{pmatrix}
\]

\item $g \circ f$ (using two-line notation)

\[
\begin{pmatrix}
2&3&5&1&4\\
4&1&2&3&5\\
\hline
4&1&2&3&5\\
2&5&3&1&4\\
\end{pmatrix}
\Rightarrow g \circ f = 
\begin{pmatrix}
2&3&5&1&4\\
2&5&3&1&4
\end{pmatrix}
\]

\item $f^{-1} \circ g^{-1}$

Obtain inverse permutations by swapping rows in two-line notation:

\begin{align*}
&f^{-1}=
\begin{pmatrix}
4&1&2&3&5\\
2&3&5&1&4
\end{pmatrix}
\quad g^{-1}=
\begin{pmatrix}
5&4&1&2&3\\
1&5&3&4&2
\end{pmatrix}\\
&\begin{pmatrix}
5&4&1&2&3\\
1&5&3&4&2\\
\hline
1&5&3&4&2\\
3&4&1&2&5
\end{pmatrix}
\Rightarrow f^{-1} \circ g^{-1} = 
\begin{pmatrix}
5&4&1&2&3\\
3&4&1&2&5
\end{pmatrix}
\end{align*}

\item $sgn(f)$

Instead of calculating $sgn(f)$ by definition we can use the fact, which was proved on seminar, that $sgn(f) = (-1)^{n-d}$ where $n$ is the number of elements in permutation and $d$ is the number of independent cycles, therefore $sgn(f) = (-1)^{5-2} = -1$

\item $sgn(g)$

$sgn(g) = (-1)^{5-1} = 1$

\item Verify that $sgn(f) \circ sgn(g) = sgn(f \circ g)$

Remember that $f \circ g = \begin{pmatrix}
1&5&3&4&2\\
2&5&3&4&1
\end{pmatrix}$ and we can factorize it on independent cycles like this: $f \circ g=(12)(5)(3)(4)$. Therefore $sgn(f \circ g) = (-1)^{5-4} = -1 = -1 \cdot 1 = sgn(f) \cdot sgn(g)$

\end{enumerate}

\item Let \[f =\begin{pmatrix}1&2&3&4&5&6&7\\4&5&7&1&3&2&6\end{pmatrix}\]
\[g = 
\begin{pmatrix}
\begin{tabular}{ccccccccccccc}
1&2&3&4&5&6&7&8&9&10&11&12&13\\
8&11&7&12&9&1&6&3&2&13&5&4&10
\end{tabular}
\end{pmatrix}
\]

\begin{enumerate}
\item Find $f^{2020}$

Split $f$ on cycles $\Rightarrow f = (14)(25376)$, then:

\[f^{2020} = (14)^{2020}(25376)^{2020}\]

We can notice that $(14)^2=id_n$ and this will be valid for any even power $(14)^{2k} = id_n \Rightarrow (14)^{2020}=id_n$.

Let's raise $(25376)$ to some powers and try to find out when we get identity permutation.  
\[(25376)^2=(23657) \quad (25376)^3=(26735) \quad (25376)^4=(27563) \quad (25376)^5=(25376)\]
So $x^5=x$, let's multiply both sides by $x^{-1}$ and get $x^4 = id_n \Rightarrow (25376)^4 = id_n$. 2020 is also divisible by 4, hence it is also an identity permutation. Combine everything together:
\[f^{2020} = (14)^{2020}(25376)^{2020} = id_n \cdot id_n = id_n\]

\item Find $g^{2077}$

Split $g$ on cycles $\Rightarrow g = (18376)(2,11,5,9)(10,13)(4,12)$, then:

\[g^{2077} = (18376)^{2077}(2,11,5,9)^{2077}(10,13)^{2077}(4,12)^{2077}\]

\end{enumerate}

\end{enumerate}

\end{document}