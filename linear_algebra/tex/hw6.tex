\documentclass{article}
\usepackage{amsmath, amssymb}
\usepackage{mathtools}

\title{DSBA Linear Algebra HW6}
\author{Kirill Korolev 203-1}
\date{8th of November, 2020}

\makeatletter
\renewcommand*\env@matrix[1][*\c@MaxMatrixCols c]{%
  \hskip -\arraycolsep
  \let\@ifnextchar\new@ifnextchar
  \array{#1}}
\makeatother

\begin{document}
\maketitle

\begin{enumerate}

\item 

\begin{enumerate}
\item What is the coefficient (in the formula for the determinant of order 7) before $a_{33}a_{16}a_{72}a_{27}a_{55}a_{61}a_{44}$?


Recall the definition of the determinant of order 7:
\[|A|=\sum_{\sigma \in S_7} sgn(\sigma) \prod_{i \in [7]} a_{i,\sigma(i)}\]

So, basically we need to find a parity of the corresponding permutation:

\begin{align*}
\sigma=
\begin{pmatrix}
3&1&7&2&5&6&4\\
3&6&2&7&5&1&4
\end{pmatrix}
\end{align*}

Rewrite it in a cycle form as $\sigma=(16)(27)$, therefore:
\[sgn(\sigma)=(-1)^{7-5}=1\]

By formula $sgn(\sigma)=(-1)^{n-d}$ where $n$ is a length of permutation and $d$ is the number of cycles.
\item Find all possible $i, j, k$ such that $a_{47}a_{63}a_{i2}a_{55}a_{7k}a_{j4}a_{31}$ occurs in the determinant of order 7 with positive sign.


Once again in order to solve this problem we need to find a parity of a permutation:
\begin{align*}
\sigma=
\begin{pmatrix}
4&6&i&5&7&j&3\\
7&3&2&5&k&4&1
\end{pmatrix}
\end{align*}

The only possible value for $k$ is $6$ and $i, j$ can be either $1, 2$ or $2, 1$, so we have the following permutations respectively: 
\begin{align*}
\sigma_1=(4763124) \quad \sigma_2=(476314)
\end{align*}

Then $sgn(\sigma_1)=(-1)^{7-2}=-1$ and $sgn(\sigma_2)=(-1)^{7-3}=1$ by the same formula.

Therefore, only $i=2, j=1, k=6$ will give a positive sign.
\end{enumerate}

\newpage
\item Transforming the matrix into Row Echelon Form, evaluate the following determinant\footnote{I hope that elementary transformations are obvious without explanations}:

\begin{align*}
&\begin{vmatrix}
2&-1&3&4\\
-2&0&-5&-1\\
4&-1&13&6\\
6&-3&14&12
\end{vmatrix}=
\begin{vmatrix}
2&-1&3&4\\
-2&0&-5&-1\\
0&1&7&-2\\
6&-3&14&12
\end{vmatrix}=
\begin{vmatrix}
2&-1&3&4\\
-2&0&-5&-1\\
0&1&7&-2\\
0&0&5&0
\end{vmatrix}=\\
&=\begin{vmatrix}
2&-1&3&4\\
0&-1&-2&3\\
0&1&7&-2\\
0&0&5&0
\end{vmatrix}=
\begin{vmatrix}
2&-1&3&4\\
0&-1&-2&3\\
0&0&5&1\\
0&0&5&0
\end{vmatrix}=
\begin{vmatrix}
2&-1&3&4\\
0&-1&-2&3\\
0&0&5&1\\
0&0&0&-1
\end{vmatrix}=2(-1)5(-1)=10
\end{align*}

\item Using the Laplace Expansion theorem, evaluate the following determinant:

\begin{align*}
&\begin{vmatrix}
5&1&2&7\\
3&0&0&2\\
1&3&4&5\\
2&1&1&-4
\end{vmatrix}=3(-1)^{2+1}\begin{vmatrix}
1&2&7\\
3&4&5\\
1&1&-4
\end{vmatrix}+2(-1)^{2+4}\begin{vmatrix}
5&1&2\\
1&3&4\\
2&1&1
\end{vmatrix}=\\
&=-3\begin{vmatrix}
1&2&7\\
0&-2&-16\\
0&-1&-11
\end{vmatrix}+2\begin{vmatrix}
0&-14&-18\\
1&3&4\\
0&-5&-7
\end{vmatrix}=-3\begin{vmatrix}
-2&-16\\
-1&-11
\end{vmatrix}-2\begin{vmatrix}
-14&-18\\
-5&-7
\end{vmatrix}=\\
&=-3(22-16)-2(98-90)=-34
\end{align*}

\newpage
\item Evaluate $det A$, where

\begin{align*}
A=
\begin{bmatrix}
k+1&k+2&k+3&\ldots{}&k+n\\
k+n+1&k+n+2&k+n+3&\ldots{}&k+2n\\
\ldots{}&\ldots{}&\ldots{}&\ldots{}&\ldots{}\\
k+(n-1)n+1&k+(n-1)n+2&k+(n-1)n+3&\ldots{}&k+n^2
\end{bmatrix}
\end{align*}

\begin{enumerate}
\item $n \geq 3$

Subtract 2nd row from the 3rd and the 1st from the 2nd one. By obtaining two same rows we can conclude that the determinant is equal to $0$.

\begin{align*}
|A|&=\begin{vmatrix}
k+1&k+2&k+3&\ldots{}&k+n\\
k+n+1&k+n+2&k+n+3&\ldots{}&k+2n\\
k+2n+1&k+2n+2&k+2n+3&\ldots{}&k+3n\\
\ldots{}&\ldots{}&\ldots{}&\ldots{}&\ldots{}\\
k+(n-1)n+1&k+(n-1)n+2&k+(n-1)n+3&\ldots{}&k+n^2
\end{vmatrix}=\\
&=\begin{vmatrix}
k+1&k+2&k+3&\ldots{}&k+n\\
k+n+1&k+n+2&k+n+3&\ldots{}&k+2n\\
n&n&n&\ldots{}&n\\
\ldots{}&\ldots{}&\ldots{}&\ldots{}&\ldots{}\\
k+(n-1)n+1&k+(n-1)n+2&k+(n-1)n+3&\ldots{}&k+n^2
\end{vmatrix}=\\
&=\begin{vmatrix}
k+1&k+2&k+3&\ldots{}&k+n\\
n&n&n&\ldots{}&n\\
n&n&n&\ldots{}&n\\
\ldots{}&\ldots{}&\ldots{}&\ldots{}&\ldots{}\\
k+(n-1)n+1&k+(n-1)n+2&k+(n-1)n+3&\ldots{}&k+n^2
\end{vmatrix}=0
\end{align*}

\item $n=1,2$

Previous transformations won't work because number of rows less than $3$.

\begin{align*}
\begin{vmatrix}
k+1
\end{vmatrix}=k+1
\end{align*}

\begin{align*}
\begin{vmatrix}
k+1&k+2\\
k+3&k+4
\end{vmatrix}=\begin{vmatrix}
k+1&1\\
k+3&1
\end{vmatrix}=k+1-k-3=-2
\end{align*}

\end{enumerate}

\end{enumerate}

\end{document}