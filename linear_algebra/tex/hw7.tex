\documentclass{article}
\usepackage{amsmath, amssymb}
\usepackage{mathtools}

\title{DSBA Linear Algebra HW7}
\author{Kirill Korolev 203-1}
\date{15th of November, 2020}

\makeatletter
\renewcommand*\env@matrix[1][*\c@MaxMatrixCols c]{%
  \hskip -\arraycolsep
  \let\@ifnextchar\new@ifnextchar
  \array{#1}}
\makeatother

\begin{document}
\maketitle

\begin{enumerate}

\item Solve the following system using Cramer’s rule:

\begin{align*}
\begin{cases}
\lambda x + 3y=\lambda - 2\\
3x+\lambda y=1
\end{cases}
\end{align*}

\begin{align*}
\Delta = \begin{vmatrix}
\lambda & 3\\
3 & \lambda
\end{vmatrix}=\lambda^2-9 \quad
\Delta_1 = \begin{vmatrix}
\lambda - 2 & 3\\
1 & \lambda
\end{vmatrix}=\lambda^2-2\lambda-3 \quad
\Delta_2 = \begin{vmatrix}
\lambda & \lambda-2\\
3 & 1
\end{vmatrix}=-2\lambda+6
\end{align*}

\begin{align*}
x = \frac{\Delta_1}{\Delta} = \frac{\lambda^2-2\lambda-3}{\lambda^2-9}=\frac{(\lambda-3)(\lambda+1)}{(\lambda-3)(\lambda+3)}
\end{align*}

\begin{align*}
y = \frac{\Delta_2}{\Delta} = \frac{-2\lambda+6}{\lambda^2-9}=\frac{-2(\lambda-3)}{(\lambda-3)(\lambda+3)}
\end{align*}

\begin{enumerate}
\item $\lambda = -3$

There are no solutions, because in this case $\Delta = 0$ and $\Delta_i \neq 0, \forall i$. Precisely, we have a contradiction $\Delta \cdot x_i = 0 \cdot x_i \neq \Delta_i$.
\item $\lambda = 3$

In this case there are infinitely many solutions, because there is one equation with two variables.
\begin{align*}
\begin{cases}
3x + 3y=1\\
3x+3y=1
\end{cases} \iff 3x + 3y=1 \iff x = \frac{1}{3} - y, \quad y \in \mathbb{R}
\end{align*}

\item In other cases there is a unique solution.
\begin{align*}
x =\frac{\lambda+1}{\lambda+3}\\
y = -\frac{2}{\lambda+3}
\end{align*}


\end{enumerate}

\newpage
\item Solve the following system using Cramer’s rule:
\begin{align*}
\begin{cases}
x+y+2z=-1\\
2x-y+2z=3\\
4x+y+4z=-3
\end{cases}
\end{align*}

\begin{align*}
\Delta &= \begin{vmatrix}
1&1&2\\
2&-1&2\\
4&1&4
\end{vmatrix}=\begin{vmatrix}
1&1&2\\
0&-3&-2\\
0&-3&-4
\end{vmatrix}=6\\
\Delta_1 &= \begin{vmatrix}
-1&1&2\\
3&-1&2\\
-3&1&4
\end{vmatrix}=\begin{vmatrix}
-1&1&2\\
0&2&8\\
0&-2&-2
\end{vmatrix}=-12\\
\Delta_2 &= \begin{vmatrix}
1&-1&2\\
2&3&2\\
4&-3&4
\end{vmatrix}=\begin{vmatrix}
1&-1&2\\
0&5&-2\\
0&1&-4
\end{vmatrix}=-18\\
\Delta_3 &= \begin{vmatrix}
1&1&-1\\
2&-1&3\\
4&1&-3
\end{vmatrix}=\begin{vmatrix}
1&1&-1\\
0&-3&5\\
0&-3&1
\end{vmatrix}=12
\end{align*}

\begin{align*}
\begin{cases}
x=\frac{\Delta_1}{\Delta}=-\frac{12}{6}=-2\\
y=\frac{\Delta_2}{\Delta}=-\frac{18}{6}=-3\\
z=\frac{\Delta_3}{\Delta}=\frac{12}{6}=2
\end{cases}
\end{align*}

\item Let $a$, $b$ and $c$ be real numbers. Then, without resorting to a direct calculation (Sarrus’ Rule is
a direct calculation!), find the value of the following determinant of matrix $X$.

\begin{align*}
|X|=\begin{vmatrix}
3&a+b+c&a^2+b^2+c^2\\
a+b+c&a^2+b^2+c^2&a^3+b^3+c^3\\
a^2+b^2+c^2&a^3+b^3+c^3&a^4+b^4+c^4
\end{vmatrix}
\end{align*}

Let A = $\begin{bmatrix}
1&a&a^2\\
1&b&b^2\\
1&c&c^2
\end{bmatrix}$. Then evidently $A^T \cdot A = X$. 

By \textbf{theorem 7.2} $|X|=|A^T|\cdot|A|$. Determinant of the Vandermonde matrix $A$ equals to $(b-a)(c-a)(c-b)$. But $|A^T|=|A|$, hence 
\[|X|=(b-a)^2(c-a)^2(c-b)^2\]

\end{enumerate}

\end{document}